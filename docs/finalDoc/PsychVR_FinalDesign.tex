% binding info: NO NOTEBOOKS OR RING BINDERS, proefssional appearance, non-paper cover, bound. 8.5x11 w/ 1 inch margins, left side 1.5in margin.

% This is a test comment. So stop reading it you piece of shit

% text formatting info: The paragraphs are to be fully justified (both left and right sides). New paragraphs may begin by indenting the line or by not indenting but leaving a space. However, DO NOT do both. The body font must be Times Roman, Arial, Helvetica, or be approved by the instructor with a font size of 10-12 pts. Heading fonts can be no larger than 20 pts. The document must be single-spaced and printing can be single or double sided. Color printing is optional and left to the discretion of the groups.

% sources info: 5. The appendix must contain written authorization (emails, letters, or explicit permission citations) for rights to include or use copyrighted content. 6. All copyrighted contents must display the content’s origin or author, and an appropriate phrase such as “reprinted with permission.” 

% citing info: 7. All elements of the document that support the written text, such as figures, table, illustrations, code segment, charts, etc., must be cited in the body of the text, and the citation must appear before the element is shown. Supportive elements cannot begin a chapter, section, or paragraph. All supportive material must be captioned, which include the element name and number and a description. If you do not author the element, the caption must contain wording identifying that you have permission to use the element. The actually permission authorization must be included in the appendices.

% reference format info: 8. Your document should contain a “References” section at the end that lists all of the source material cited in your document. The references should be listed in the order that they first appear in your document in a format like the following (first is a book, second is an article): [1] J. Hennessy and D. Patterson. Computer Architecture: A Quantitative Approach, 5 th edition, Morgan Kaufman publishers, 2014. [2] M. Heinrich et al. “The Performance Impact of Flexibility in the Stanford FLASH Mujltiprocessor”, In Proceedings of the 6th International Conference on Architectural Support for Programming Languages and Operating Systems (ASPLOS), pages 274-285, 1994

% citation formatting: When citing the references in the text use a non-breakable space (e.g. Ctrl-shift-space in Word, \&; in HTML) before the citation and then square brackets and the number: Some researchers say that the occupancy of the node controller is the parameter most critical to the performance of the overall machine [2]. If you want to cite multiple references at once just separate them by commas like so [1,3,7].

\documentclass[a4paper,10pt]{article}
\usepackage[utf8x]{inputenc} 
\usepackage{graphicx}
\usepackage{float}
\usepackage{listings}
\usepackage{geometry}
	\geometry{
	letterpaper,
	total={8.5in,11in},
	left=1.5in,
	top=1in,
	right=1in,
	bottom=1in
}

%opening
\pagenumbering{roman}
%1. Cover page with title, group number, team members, date, and any other relevant information, such as participating organizations and sponsors.
\title{Group 12 Project Final Design Document}
\author{Eric Momper, Peter Lomason, John Barber}
\setlength{\parindent}{0pt}
\begin{document}
	
\maketitle
	
	\pagebreak
	\tableofcontents
	\pagebreak
	
	\section{Introduction}
		Our Project is intended to be a psychological therapeutic tool that helps patients through simulation in Virtual Reality.
	Many of the new up and coming Virtual Reality devices look very promising at providing better and more realistic immersion and simulation for users at a lower cost than ever before. We aim to bring psychological tools to the home of the average user. This will allow those seeking certain therapies or treatments to perform them more often since such services often require high diligence and repetition to produce results.
	
	\paragraph{A few of these devices include:}
	\begin{itemize}
		\item Oculus Rift, Oculus Touch
		\item HTC Vive
		\item Samsung Gear VR
	\end{itemize}
	
	\paragraph{ Some of our project ideas are related to the following:}
	\begin{itemize}
		\item ​Immersion Therapy for phobias, dysphorias, or PTSD (Virtual or Augmented Reality)
		\item Therapy for burn victims, phantom pain for amputees,  (Virtual Reality)
		\item Creating a calm environment for  anxiety disorders, or autism (Virtual Reality)
		\item Creating a drawing art therapy tool that allows for creativity in 3D space (Virtual or Augmented Reality)
	\end{itemize}
	
	
	\paragraph{Project Direction} ~\\ We are currently contacting faculty in the psychology department for collaboration or guidance on opportunities to use our project for helping their patients or for their research.
	\paragraph{Integration} ~\\ Our project will be created in Unity, Unreal Engine, or written with a C++ OpenGL/DirectX SDK wrapper, depending on the direction of our design, and the VR device we have available. It will most likely be windows exclusive as many VR devices are dropping Linux and OSX support unless we can find some options that will provide cross platform systems, which is said to be available in the near future for some platforms but not currently finished.  
	
	\pagebreak
	
	\section{Status of Virtual Reality in Psychology}
	The first use of Virtual Reality therapy in Psychology was in 1995 by psychologist Barbara Rothbaum and computer scientist Larry Hodges. T
	hey found that virtual reality therapy could help patients overcome phobias such as arachnophobia or a fear of heights. Since then many others 
	have used virtual reality as a tool in psychology. The main use of virtual reality in psychology is a form of treatment called Exposure Therapy. 
	This type of treatment can be used to address psychological issues such as Autism spectrum disorder, Obsessive Compulsive Disorder, various phobias, 
	post-traumatic stress disorder, and phantom limb pain. The greatest issue facing treatment with Exposure Therapy is that it requires a high level of diligence and
	repetition. Typically patients cannot make time for appointments as frequently as is required.
	
	\subsection{Immersion Therapy}
	This psychological treatment helps patients simulate past or hypothetical events so they can adapt to or reason through
	various situations. Our particular area of focus is fear of heights (Acrophobia).
	\subsection{Pain Treatment}
	Some examples of this are used with burn victims or people with acute pain, putting them in a calming or cold environment to distract them to relieve pain.
	\subsection{Creating a Calming Environment}
	This can be useful for many patients with anxiety disorders as it will allow users to be in a relaxing environment with softer stimuli that can take their mind off of anxiety issues. 
	\pagebreak
	
	
	\section{Team Member Motivations}
	\subsection{Eric Momper}
	My personal motivation for this project is that it is similar to some of the programming I do on my internship (OpenGL and OpenCL image GPGPU processing) or graphics.
	I also have taken computer graphics with Professor Leinecker and I am currently taking robot vision with Doctor Lobo. This project will be very interesting to me as  
	I will be working with new technologies and programming on a new type of 3D graphics platform. I am also looking forward to studying the psychological benefits
	of Virtual Reality on patients with various disorders. My focus in the project will be on the Qt configuration tool as I have experience with Qt and C++, I will 
	also work on the configurable parts of the Unity scene and will assist with basic scene design, scripting. We may also use my computer or other parts of hardware for more powerful headset options, 
	as my GPU/CPU AMD R9 290 / I7 4790k/ 16GB DDR3 2133mhz can run most VR systems at a high resolution and framerate. 
	
	\subsection{John Barber}
	This project was my idea, and combines two different fields, virtual reality and psychology.  In the same way, i'm interested in both 
	sides separately, and really hopeful about how they can be combined.  Virtual Reality is exciting to me as a game designer, 
	and one of my closest friends went to work for Emblematic Group, and we've been comparing notes on the future of virtual reality since.  
	However, I also have seen a number of my friends struggle with psychological issues, and have been hoping to find a way to use my Computer 
	Science degree after I graduate to help people.  This is an opportunity to not only accomplish that now, pushing the field forward and finding 
	new ways to use it for people, but also to train myself and find opportunities and connections for the future.
	\subsection{Peter Lomason}
	My interest in this project mainly comes from past experience with virtual reality tools. I used to own an Oculus Rift DK1 and at the time I had it, 
	I was not knowledgeable enough to develop for it. Now I would like to apply what I have learned at UCF to virtual reality development. I eventually want to be
	a video game developer and virtual reality for psychology shares many aspects with that. Being able to program an environment, objects, and interactions is very
	interesting to me and this project will allow me to strengthen my ability to do these tasks.
\pagebreak



\pagenumbering{arabic}

%Project Documentation Guidelines
%additions / deletions where appropriate

%2. Executive summary: An administrative and technical abstract, which includes a brief description of the project, the project objectives, and the technical approach. This is really an overview of 3A, 3B, 3C, and 3D. This is page number 1.
\section{Executive Summary}
 The relationship between psychology, therapy, and virtual reality is no new concept. Benefits which can be derived through the use of virtual reality as a treatment option are vast 
 and already well researched with many successful studies and experiments supporting this type of therapy. A caveat to this success is that most of the treatment is carried out as research, 
 at limited locations, or on a very exclusive basis.
~\\ ~\\
 This is due to a number of factors including the available openings for research, cost of the hardware and software involved, and locations
 of the few companies which professionally offer virtual reality therapy. Psychological and social factors also play a role in the underwhelming prevalence of virtual reality for psychology for 
 reasons PsychVR aims to remedy this problem by offering a suite of treatment environments targeted at some of the most prominent psychological issues which respond well to virtual reality therapy.
 ~\\ ~\\
 The first three environments to be developed will target fear of heights, speech anxiety, and stress relief. Our platform will feature customization of these environments and expandability for more.
 To tackle the issue of hardware cost to the average user it will be available for a variety of virtual reality headsets and controls, most notably including Samsung Gear VR. It is projected that the 
 Gear VR will soon be the most widely available virtual reality headset at a cost of 100\$ along with a compatible Samsung smartphone. By developing the suite in Unity for the Oculus Rift, the end result
 will be compatible with the Gear VR which uses the same framework as the Rift.
\pagebreak
%3. Technical content (This is NOT an outline, just a list of what needs to be included)
%A. Identification of the project and its significance, motivation, etc. (mostly text). Please include personal motivation statements for each project member here. Also you MUST include a separately labeled section or subsection with the title “Broader Impacts” that describes in a minimum of 1 paragraph, the broader implications or impact of your project on society both local and global. How might your project impact underrepresented groups (within science and technology (STEM) or society as a whole), the disabled, non-profit organizations, the environment, diversity, increased participation in STEM fields or the workforce, public engagement in STEM areas, improved national security, enhanced infrastructure, or improved education are all examples.
%B. Technical objectives, goals, specifications, and requirements (mostly text and numbers) FIRST PAPER HERE EZPZ
\section{Project Goals and Objectives}
	Design an environment in Virtual Reality that can be customized and used to help a variety of psychological conditions, and further psychology research, that is accessible to the typical user so they may conduct their own treatment on their own time.
	\begin{itemize}
		\item Increase our understanding of psychology principals \& problems and how virtual reality can help certain conditions.
		\item Find a platform that we can develop on, that creates a high quality virtual reality experience, and is reasonably up to date with modern graphics. Currently our best candidate is the Unity Engine.
		\item Integrate some level of user intractability in the created virtual environments. 
		\item Procedural generation of environments with parameters that can be customized by the user. 
		\item Include pre-constructed environments similar to current treatment strategies in the VR Psychology industry.
	\end{itemize}
	\section{Broader Impacts}
	%Broad implications and impact on society (impact on underrepresented? within STEM and or society as a whole? disabled? non-profit orgs? environment? diversity? increased participation in STEM or workforce? public engagement in STEM? improve national security? enhanced infrastructure? improved education?)
	
	%qualitative, avoid numbers, conceptual discussion specific to project. example descriptions "“lightweight, portable, programmable, low cost, flexible, high resolution, scalable, low power, accurate, mobile, peer-to-peer, autonomic”
	
	Our project would have an impact on the field of psychology as it relates to certain emotional disorders or struggles and the deployment of self-administered therapy for those disorders. It would not only be relevant to people diagnosed with a psychological condition, but also those seeking stress relief or a unique environment to immerse themselves in. By bringing treatment therapies to the end user in their own home and allowing them to perform therapy at their leisure we will have an impact on many people seeking these services who may not have the time to schedule appointments or those who simply won't try due to the stigma surrounding therapy.
	
	
	
	\pagebreak
%D. Detailed design reqs add more here...
\section{Specifications and Requirements}
	\subsection{Functional Requirements}
	\begin{enumerate}
		\item Interactive Environments
		\begin{itemize}
		 \item Description: The user should be presented with a realistic interactive 3D Environment. 
		 \item Dependency: Assets should be high quality, and the Unity scenes should run well.
		 \item Evaluation: Testing scenes by using them with VR.
		\end{itemize}
		\item Psychological Treatment
		\begin{itemize}
		 \item Description: The scenes should be useful for some sort of psychological testing or treatment.
		 \item Dependency:  Psychology professors or UCF staff should be consulted for approval and input.
		 \item Evaluation:  Approval from professors. 
		\end{itemize}
		\item Configurable Environments
		\begin{itemize} 
		 \item Description: Scenes should be configurable with the use of an external tool, that should change the environment. 
		 \item Dependency:  Scene loads assets or changes layouts from Qt GUI configuration app's config file. 
		 \item Evaluation:  Test a configurable scene.
		\end{itemize}
		\item Realistic User Interactions
		\begin{itemize}
		 \item Description: User controls should be simple and feel realistic. 
		 \item Dependency:  Dealing with latency and processing of user input in Unity. 
		 \item Evaluation:  Testing scenes with controls.
		\end{itemize}
		\item Qt Environment maps
		\begin{itemize}
		 \item Description: The Qt configuration tool should have drag and drop maps that allow the user to change the layout of configured maps.
		 \item Dependency:  Qt drag and drop and OpenGL drawing widgets. 
		 \item Evaluation:  Testing this feature.
		\end{itemize}
		\item Qt Object Lists
		\begin{itemize}
		 \item Description: The Qt app should have a list of available objects to be placed or moved in the Scene, (this can be done by another configuration tool 
		 (Creator module) or be a hard coded list of assets linked with the scene), this will need to be stored in a different config file. 
		 \item Dependency:  The environment map needs a list of Objects to configure. 
		 \item Evaluation:  Try this feature with a list of objects.
		\end{itemize}

		\item Qt Usability
		\begin{itemize}
		 \item Description: The Qt Desktop app should be usable at various resolutions with easy simple controls. 
		 \item Dependency:  Making the tool usable.
		 \item Evaluation:  Test the tool, make sure it is easy to use.
		\end{itemize}
		
		\item Qt Configuration Files
		\begin{itemize}
		 \item Description: Scene layouts and other details related to configuration of the scenes must be saved in a file to be read in by Unity.
		 \item Dependency:  Making the scenes configurable.
		 \item Evaluation:  Save map editor configurations and read them in in Unity.
		\end{itemize}
		\item Qt Portability (If Configuration is supported on Android)
		\begin{itemize}
		 \item Description: If we support configurable scenes on android, we need to build and android version of the Qt app, this may already be supported by Qt, as it is highly portable.
		 \item Dependency:  Portability of the configuration tool.
		 \item Evaluation:  Build the Qt app to Android, or write another one. 
		\end{itemize}
		\textbf{ Now that these basic functional requirements have been defined, specific module requirements are as follows:}
		
		\item Scene 1 Fear of Heights Movement 
		\begin{itemize}
		 \item Description: The scene shall use some sort of movement tracking camera to create deeper immersion for the user, (ex: Xbox Kinect or DK2+ IR camera).
		 \item Dependency: Usability of the scene.
		 \item Evaluation: Test the scene with subjects and show to Psychology department. 
		\end{itemize}
		
		\item Scene 1 Fear of Heights Interaction 
		\begin{itemize}
		 \item Description: The user shall be put into some sort of simulated high up place with configurable scenery, they should be able to move around and interact with the scene.
		 \item Dependency: Usability of the scene.
		 \item Evaluation: Test the scene with subjects and show to Psychology department. 
		\end{itemize}
		
		
		\item Scene 1 Fear of Heights Objective 
		\begin{itemize}
		 \item Description: The user shall be put into some sort of simulated high up place with configurable scenery, this scene shall have some sort of objective (walk to point, stand for so long etc.)
		 \item Dependency: Giving the user some realistic simulation with an objective (metric).
		 \item Evaluation: Test the scene with subjects and show to Psychology department. 
		\end{itemize}
		
		\item Scene 2 Speech Anxiety Interaction
		\begin{itemize}
		 \item Description: The user shall be put into some sort of simulated high up place with configurable scenery, they should be able to move around and interact with the scene and people in it.
		 \item Dependency: Usability of the scene.
		 \item Evaluation: Test the scene with subjects and show to Psychology department. 
		\end{itemize}
		\item Scene 1 Fear of Heights Objective 
		\begin{itemize}
		 \item Description: The user shall have some sort of objective, or metric to measure speech improvement, (overall feeling, mannerisms, different observations)
		 \item Dependency: Giving the user some realistic simulation with an objective (metric).
		 \item Evaluation: Test the scene with subjects and show to Psychology department. 
		\end{itemize}
		
		
		\item Scene 3 Calming Environment Interactions
		\begin{itemize}
		 \item Description: The user shall be put into some sort of simulated high up place with configurable scenery, they should be able to move around and interact with the scene.
		 \item Dependency: Usability of the scene.
		 \item Evaluation: Test the scene with subjects and show to Psychology department. 
		\end{itemize}
		
		\item Scene 3 Calming Environment Interactions 
		\begin{itemize}
		 \item Description: The scene shall have configurable calming sounds and visuals
		 \item Dependency: Calming art/ experience for the user.
		 \item Evaluation: Test the scene with subjects and show to Psychology department. 
		\end{itemize}
		
		\item Scene 3 Calming Environment Terrain generation
		\begin{itemize}
		 \item Description: The scene shall have some sort of procedurally generated terrain. 
		 \item Dependency: Large environments for the user to explore, creating a more indepth experience. Probably using computer shaders that are supported in Unity!!!
		 \item Evaluation: Test the scene with subjects and show to Psychology department. 
		\end{itemize}
		
	\end{enumerate}

	\subsection{System Requirements}
	\subsubsection{PC}
		\begin{enumerate}
			\item Hardware Minimum Spec for Running VR
			\begin{itemize}
				\item Description: Quality and frame rate can vary but must be at least 30fps.
				\item Dependency: Minimum supported GPU, CPU, RAM, OS Specs (See Specific Headset Research 9.x.x) 
				\item Evaluation Method: Test app on Real Hardware (VR Headset, with Controls)
			\end{itemize}
			\item Low Latency Controls
			\begin{itemize}
				\item Description: Minimum system requirements to Peripherial input processing
				\item Dependency: Minimum supported GPU, CPU, RAM, OS Specs (See Specific Peripherial Research 9.x.x) 
				\item Evaluation Method: Test app on Real Hardware (VR Headset, with Controls)
			\end{itemize}
		\end{enumerate}
	\subsubsection{Android}
		\begin{enumerate}
			\item Hardware Minimum Spec for Running Smooth VR
			\begin{itemize}
				\item Description: It should operate between 30 and 60fps depending on CPU load.
				\item Dependency: Capped at 60fps due to hardware limitations (See Gear VR 8.2.4)
				\item Evaluation Method: Test app on Real Hardware (Gear VR Headset, with Controls)
			\end{itemize}
			\item Low Latency Controls
			\begin{itemize}
				\item Description: Minimum system requirements to Peripheral input processing
				\item Dependency: Minimum supported GPU, CPU, RAM, OS Specs (See Specific Peripheral Research 9.x.x) 
				\item Evaluation Method: Test app on Real Hardware (VR Headset, with Controls)
			\end{itemize}
			\item Relevant Controls
			\begin{itemize}
				\item Description: Scenes should only require a controller when necessary.
				\item Dependency: Level of user interaction with environment determines input style.
				\item Evaluation Method: Appropriate controls will make the user more immersed in the experience.
			\end{itemize}
		\end{enumerate}
	
	\subsection{Resource Requirements}
	\begin{enumerate}
	 \item Software tools
	 \begin{itemize}
	 \item Description: The tools that will be used for development of the Desktop and Android app  include the Unity Engine development platform, an Android phone (Samsung S6 Edge) and emulator for running/ testing the Unity App,
	  along with the Qt Tool kit for the configuration tool.
	 \item  Dependency: Necessary tools for development are free, and easy to use. 
	 \item Evaluation Method: Make sure all team members are able to use and access development tools/
	  \end{itemize}
	 \item Media Assets
	\begin{itemize}
	  \item Description: Game Character models need to be designed or purchased. Autodesk Maya, Google Sketchup or blender should be used for designing additional models. 
	  \item Dependency: Creating realistic visuals 
	  \item Evaluation Method: Test app to ensure it performs well and represents a high quality environment.
	 \end{itemize}
	\end{enumerate}

	\subsection{Data Requirements}
	\begin{enumerate}
	 \item Saving Scene configurations
	 \item Placing configured scenes With accurate locations
	\end{enumerate}

	\subsection{Human Factors Requirements}
	\begin{enumerate}
	 \item UI Simplicity
	 \item VR Sickness	 
	 \item The models, textures, and graphics in the environment must be decent quality to help immerse the user.
	 \end{enumerate}
	\subsection{Documentation Requirements}
	\begin{enumerate}
	 \item Clear Simple User Guides
	 \item Accessibility of Guides
	\end{enumerate}
	\subsection{Quality Assurance Requirements}

	
	
\pagebreak
%C. Research and investigations (text, numbers, tables, charts, figures, diagrams)
\section{Research}
Research for our project breaks up into a few main areas:
\begin{itemize}
\item Virtual Reality Device, Platform, and Market Research
	\begin{itemize}
	\item Growth, Costs, and projections of popular devices
	\item History of Companies
	\item Technical Capabilities of Headsets
	\item Peripheral Options
	\end{itemize}
\item Psychology Usage of VR
\item Psychology research on specific conditions we want to treat
	\begin{itemize}
	\item Fear of Heights
	\item Speech Anxiety
	\item Calm Environment
	\end{itemize}
\item Modular Templating of Unity Scenes
\item Peripheral / User Input / Sensor Options
\end{itemize}
\pagebreak
\subsection{VR Market Research}
market stuff
\pagebreak

\subsection{VR Headset Options}

\subsubsection{Oculus Rift Consumer Version}
\begin{itemize}
	 \item Resolution: 2160 x 1200
	 \item Refresh Rate: 90 Hz (11 ms !!)
	 \item Latency: 20ms
	 \item Recommended CPU: Intel i5-4590 or equivalent
	 \item Recommended GPU: NVIDIA GTX 970 / AMD 290 
	 \item Recommended RAM: 8GB
	 \item Positional Tracking: IR Camera Sensor (5 x 11 feet)
	 \item Controls: Oculus Touch (not included)
	 \item Release Date: Q1 2016 (Delayed)
	 \item Manufacturer: Oculus VR
	 \item Unity Support: Yes
	 \item OS Cross Platform: Windows and Android only. (Oculus Home Dependent)
	 \item Cost: \$599
\end{itemize}
\begin{figure}[H]
	\includegraphics[width=\linewidth,height=\paperheight,keepaspectratio]{cv.jpg}
	\caption{Oculus Rift Consumer Version}
	%to ref fig number
	%Figure \ref{fig:block1} shows our blockDiagram.
	\label{fig:RiftCVImg}
	\end{figure}
	\pagebreak
\subsubsection{Oculus Rift Dev Kti 2}
\begin{itemize}
	 \item Resolution: 1920 x 1080
	 \item Refresh Rate: 75 Hz (13 ms)
	 \item Latency: 20-40ms
	 \item Recommended CPU: Intel i5-4590 or equivalent
	 \item Recommended GPU: NVIDIA GTX 970 / AMD 290 
	 \item Recommended RAM: 8GB
	 \item Positional Tracking: IR Camera Sensor (5 x 11 feet)
	 \item Controls: Oculus Touch (not included)
	 \item Release Date: July 2014
	 \item Manufacturer: Oculus VR
	 \item OS Cross Platform: Windows and Android only. (Oculus Home Dependent)
	 \item Unity Support: Yes
	 \item Cost: \$350
\end{itemize}
\begin{figure}[H]
	\includegraphics[width=\linewidth,height=\paperheight,keepaspectratio]{dk2.jpg}
	\caption{Oculus Rift Dev Kti 2}
	%to ref fig number
	%Figure \ref{fig:block1} shows our blockDiagram.
	\label{fig:Riftdk2Img}
	\end{figure}
	\pagebreak
\subsubsection{Oculus Rift Dev Kit 1}
\begin{itemize}
	 \item Resolution: 1200 x 800
	 \item Refresh Rate: 60 Hz (16 ms)
	 \item Latency: 50-60ms
	 \item Recommended CPU: Lower Spec (Early Prototype)
	 \item Recommended GPU: Lower Spec (Early Prototype)
	 \item Recommended RAM: 4GB
	 \item Positional Tracking: No
	 \item Controls: Oculus Touch (not included) 
	 \item Release Date: August 2012 (Kickstarter)
	 \item Manufacturer: Oculus VR
	 \item Cross Platform: Windows and Android only. (Oculus Home Dependent)
	 \item Unity Support: Yes
	 \item Cost: \$300
\end{itemize}
	\begin{figure}[H]
	\includegraphics[width=\linewidth,height=\paperheight,keepaspectratio]{dk1.jpg}
	\caption{Oculus Rift Dev Kit 1}
	%to ref fig number
	%Figure \ref{fig:block1} shows our blockDiagram.
	\label{fig:Riftdk1Img}
	\end{figure}
	\pagebreak
\subsubsection{Samsung Gear VR}
	Supported Phones Include the Samsung Galaxy: Note5, S6, S6 edge, S7, S7 edge  
	\begin{itemize}
	  \item Resolution: 2560 X 1440 (Quad HD phone screen)
	  \item Refresh Rate: 60 Hz (16 ms)
	  \item Latency: 50-60ms
	  \item Recommended CPU: Phone Dependent
	  \item Recommended GPU: Phone Dependent
	  \item Recommended RAM: Phone Dependent
	  \item Positional Tracking: No
	  \item Controls: Touch pad, back button, volume controls
	  \item Release Date: November 2015
	  \item Manufacturer: Samsung, With technology by Oculus VR
	  \item Cross Platform: Windows and Android only. (Oculus Home Dependent)
	  \item Unity Support: Yes
	  \item Cost: \$99
	\end{itemize}
	\begin{figure}[H]
	\includegraphics[width=\linewidth,height=\paperheight,keepaspectratio]{gear.jpg}
	\caption{Samsung Gear VR}
	%to ref fig number
	%Figure \ref{fig:block1} shows our blockDiagram.
	\label{fig:GearImg}
	\end{figure}
	\pagebreak
\subsubsection{HTC Vive}
	\begin{itemize}
	  \item Resolution: 2160 x 1200
	  \item Refresh Rate: 90 Hz (11 ms!!!)
	  \item Latency: 22ms
	  \item Recommended CPU: Intel i5-4590 or equivalent
	  \item Recommended GPU: NVIDIA GTX 970 / AMD 280 
	  \item Recommended RAM: 4GB
	  \item Positional Tracking: IR Camera Sensor (15 x 15 feet)
	  \item Controls: Motion Controllers (included) (similar to Occulus Touch)  
	  \item Release Date: Q1 2016 (Delayed)
	  \item Manufacturer: HTC, With technology by Valve Corporation
	  \item Cross Platfrom: Valve OpenGL Mesa support/ SteamVR support on Linux soon hopefully
	  \item Unity Support: Yes
	  \item Cost: \$799
	\end{itemize}
	\begin{figure}[H]
	\includegraphics[width=\linewidth,height=\paperheight,keepaspectratio]{vive.jpg}
	\caption{HTC Vive Headset}
	%to ref fig number
	%Figure \ref{fig:block1} shows our blockDiagram.
	\label{fig:ViveImg}
	\end{figure}
	\pagebreak
\subsubsection{PS4 Morpheus}
\begin{itemize}
  \item Resolution: 3840 x 1080 
  \item Refresh Rate: 120 Hz (8 ms!!!)
  \item Latency: 18ms
  \item Recommended CPU: PS4
  \item Recommended GPU: PS4
  \item Recommended RAM: PS4
  \item Positional Tracking: Playstation Camera area ~(18 x 12 feet)
  \item Controls: Playstation Move Controllers (not included)
  \item Release Date: Q1 2016 (Delayed)
  \item Manufacturer: Sony
  \item Cross Platfrom: No
    \item Unity Support: Yes
  \item Unity Support: Yes
  \item Cost: \$399
\end{itemize}
\begin{figure}[H]
	\includegraphics[width=\linewidth,height=\paperheight,keepaspectratio]{morpheus.jpg}
	\caption{Playstation VR}
	%to ref fig number
	%Figure \ref{fig:block1} shows our blockDiagram.
	\label{fig:psvrImg}
	\end{figure}
	\pagebreak
	
\pagebreak
\subsection{VR Peripheral Options}
precision, inputs, basic info
\subsubsection{Leap Motion}
	Using Leap Motion on a PC is viable in Unity and very well supported, however using Leap Motion and the Gear VR simultaneously on Android is not feasible.
\subsubsection{Oculus Rift Move}
\subsubsection{Red Samurai Bluetooth Controller}
	A cheap, 8\$ bluetooth controller, that pairs very well with the Gear VR on Android.
\subsubsection{Gear VR Peripherals?}
\pagebreak
\subsection{Movement Tracking Cameras}
\subsubsection{Microsoft Kinect 1}
\begin{itemize}
  \item Resolution: 640 X 480
  \item Refresh Rate: 33 Hz (30 FPS)
  \item Max Depth: 12 feet
  \item Release Date: August 2010 
  \item Manufacturer: Microsoft
  \item Cross Platform: Yes
  \item Unity Support: Yes
  \item Skeleton Joints:: 26
  \item Horizontal FOV: 57 Degrees
  \item Vertical FOV: 43 Degrees
  \item Pixels Per Degree: 5 X 5
  \item Technique: Structured Light Patterns
  \item Cost: Acquired, Not widely available, ~\$40
\end{itemize}
\begin{figure}[H]
	\includegraphics[width=\linewidth,height=\paperheight,keepaspectratio]{kinect1.jpg}
	\caption{Kincet 1 Camera}
	%to ref fig number
	%Figure \ref{fig:block1} shows our blockDiagram.
	\label{fig:k1Cam}
	\end{figure}
	\pagebreak
	\subsubsection{Microsoft Kinect 2}
\begin{itemize}
  \item Resolution: 1920 X 1080
  \item Refresh Rate: 33 Hz (30 FPS)
  \item Max Depth: 12 feet
  \item Release Date: August 2010 
  \item Manufacturer: Microsoft
  \item Cross Platfrom: Yes
  \item Skeleton Joints: 26
  \item Horizontal FOV: 70 Degrees
  \item Vertical FOV: 60 Degrees
  \item Pixels Per Degree: 7 X 7
  \item Technique: Time of Flight
  \item Cost: \$99
\end{itemize}
\begin{figure}[H]
	\includegraphics[width=\linewidth,height=\paperheight,keepaspectratio]{kinect2.jpg}
	\caption{Kincet 2 Camera}
	%to ref fig number
	%Figure \ref{fig:block1} shows our blockDiagram.
	\label{fig:k2Cam}
	\end{figure}
	\pagebreak
	
\subsubsection{Comparison of Kinect Systems}
\textbf{(Figure \ref{fig:kimg})} shows the increased fidelity and quality of the captured image on the kinect 2.
\begin{figure}[H]
	\includegraphics[width=\linewidth,height=\paperheight,keepaspectratio]{kinectImg.jpg}
	\caption{Kincet 2 Camera}
	%to ref fig number
	%Figure \ref{fig:block1} shows our blockDiagram.
	\label{fig:kimg}
	\end{figure}
	\pagebreak	
	
\subsubsection{Oculus Rift DK2+ Camera}
\begin{itemize}
  \item Resolution: 752×480
  \item Refresh Rate: 16 Hz (60 FPS)
  \item Max Depth: ?
  \item Release Date: July 2014
  \item Manufacturer: Oculus Rift
  \item Cross Platfrom: No
  \item Skeleton Joints: Heaset Position
  \item Horizontal FOV: ?
  \item Vertical FOV: ?
  \item Pixels Per Degree: ?
  \item Technique: IR LED Pattern Recognition
  \item Cost: Included With Headset
\end{itemize}
 \begin{figure}[H]
	\includegraphics[width=\linewidth,height=\paperheight,keepaspectratio]{riftIR.jpg}
	\caption{Rift Sensor IR}
	%to ref fig number
	%Figure \ref{fig:block1} shows our blockDiagram.
	\label{fig:riftCam}
	\end{figure}
	\pagebreak
	\subsubsection{HTC Vive Lighth House Sensors}
\begin{itemize}
  \item Resolution: 752×480
  \item Refresh Rate: 16 Hz (60 FPS)
  \item Max Depth: ?
  \item Release Date: July 2014
  \item Manufacturer: Oculus Rift
  \item Cross Platfrom: No
  \item Skeleton Joints: Heaset Position
  \item Horizontal FOV: NA
  \item Vertical FOV: NA
  \item Pixels Per Degree: ?
  \item Technique: 360 Laser / LED full room scanning 
  \item Cost: Included With Headset
\end{itemize}
 \begin{figure}[H]
	\includegraphics[width=\linewidth,height=\paperheight,keepaspectratio]{viveLight.jpg}
	\caption{HTC Lighthouse Sensor}
	%to ref fig number
	%Figure \ref{fig:block1} shows our blockDiagram.
	\label{fig:viveCam}
	\end{figure}
	\pagebreak

\subsection{Compnay Histories}
%rift startup, other infos here...
\subsection {Psychology research}
This Section will discuss various psychological areas of our project including warnings, limitations, trends, other successful tests, and specifics about different phobias or disorders 
that we are researching. 

\paragraph{DSM Cautionary Statement}  ~\\
Any cited DSM excepts ofter include this statement as a disclaimer. \cite{dsmCaution}
\begin{itemize}

\item The specified diagnostic criteria for each mental disorder are offered as guidelines for making diagnoses, because it has been demonstrated that 
the use of such criteria enhances agreement among clinicians and investigators. The proper use of these criteria requires specialized clinical
training that provides both a body of knowledge and clinical skills. 

\item These diagnostic criteria and the DSM-IV Classification of mental disorders reflect a consensus of current formulations of evolving knowledge in
our field. They do not encompass, however, all the conditions for which people may be treated or that may be appropriate topics for research efforts. 

\item The purpose of DSM-IV is to provide clear descriptions of diagnostic categories in order to enable clinicians and investigators to diagnose, communicate
about, study, and treat people with various mental disorders. It is to be understood that inclusion here, for clinical and research purposes, of a diagnostic
category such as Pathological Gambling or Pedophilia does not imply that the condition meets legal or other nonmedical criteria for what constitutes mental disease,
mental disorder, or mental disability. The clinical and scientific considerations involved in categorization of these conditions as mental disorders may not be wholly 
relevant to legal judgments, for example, that take into account such issues as individual responsibility, disability determination, and competency.
\end{itemize}
\pagebreak
\subsubsection{DSM Criteria for Specific Phobias}
\paragraph{DSM IV}~\\
(cautionary statement) 
\begin{itemize}
 \item Marked and persistent fear that is excessive or unreasonable, cued by the presence or anticipation of a specific object or situation (e.g., flying, heights, animals, receiving an
 injection, seeing blood). 
 \item Exposure to the phobic stimulus almost invariably provokes an immediate anxiety response, which may take the form of a situationally bound or situationally predisposed Panic Attack. 
 Note: In children, the anxiety may be expressed by crying, tantrums, freezing, or clinging. 
\item  The person recognizes that the fear is excessive or unreasonable. Note: In children, this feature may be absent. 
\item The phobic situation(s) is avoided or else is endured with intense anxiety or distress. 
\item The avoidance, anxious anticipation, or distress in the feared situation(s) interferes significantly with the person's normal routine, occupational (or academic) functioning, or
social activities or relationships, or there is marked distress about having the phobia. 
\item In individuals under age 18 years, the duration is at least 6 months.
\item The anxiety, Panic Attacks, or phobic avoidance associated with the specific object or situation are not better accounted for by another mental disorder, such as Obsessive-
Compulsive Disorder (e.g., fear of dirt in someone with an obsession about contamination), Posttraumatic Stress Disorder (e.g., avoidance of stimuli associated with a severe stressor),
Separation Anxiety Disorder (e.g., avoidance of school), Social Phobia (e.g., avoidance of social situations because of fear of embarrassment), Panic Disorder with Agoraphobia, or 
Agoraphobia Without History of Panic Disorder. \cite{dsmPhobia}
\end{itemize}
 
\paragraph{Specific types:} 
\begin{itemize}
 \item Animal Type
 \item Natural Environment Type (e.g., heights, storms, water) 
 \item Blood-Injection-Injury Type 
 \item Situational Type (e.g., airplanes, elevators, enclosed places) 
 \item Other Type (e.g., phobic avoidance of situations that may lead to choking, vomiting, or contracting an illness; in children, avoidance of loud sounds or costumed characters)
\end{itemize}
\pagebreak
\subsubsection{Examples of Success}
For example, 12 studies tested the effects of virtual reality on stroke recovery. 5 of these studies showed patients who played virtual reality games were about 4.9 times more likely to improve their upper body strength than those who received the standard therapy, while the other 7  studies showed a 14.7 percent improvement, on average, in patients' grip strength and a 20 percent improvement in patient's ability to perform standard tasks.\cite{stroke1}
\pagebreak
\subsection{Module 1 Fear of Heights}
research here
\pagebreak
\subsection{Module 2 Speech Anxiety Simulator}
research for this
\pagebreak
\subsection{Module 3 Calm Environment} % SO HYPED!!!
This Module will feature a calming environment that will help patients with stress or anxiety relax in a calming peaceful environment. This environment will have procedurally generated terrain which will have chunks 
that are procedurally while the user moves throughout the scene. Some solutions include some premade plug-ins that generate some terrains, on the CPU with possibly some GPU acceleration, however we would like to try and generate our own
terrain with Compute Shader provided in Unity. Some factors that may limit out ability to do this include access to some features related to Direct render buffers which may only come with the Professional Version, 
along with the efficiency of our generation and mapping algorithms on the GPU which will already have the heavy load of meeting rendering demands of release level VR headsets which require a large frame buffer for 
two screens which each require different OpenGL rendering passes, most of this will be handled by Unity for most scenes but if we want to do any mapping on the fly we will have to meet these rendering/ processing 
time lines for our compute operations, and deal with the extra processing overhead for the GPU while making sure we can run the VR with decent settings and frame rates. Unity HLSL 
    % CODE!!! \lstinputlisting [language =C] {test.hlsl} 

% https://scrawkblog.com/category/directcompute/
% http://answers.unity3d.com/questions/162096/gpu-programming-with-unity.html
\pagebreak
\subsubsection{General-purpose computing on graphics processing units (GPGPU)}~\\
GPGPU programming is a growing option for parallel processing applications that work over large data sets. Some common uses include image or media processing and 
performing mathematical computations over linear systems or large sets. Many modern GPUs offer many benefits in processing with the high number of cores that can operate on data 
in Single Instruction Multiple Data (SIMD) and Multiple Instruction Multiple Data (MIMD) fashions. For example, one group of data can be processed with the same instruction in parallel 
and the GPU can have multiple groups scheduled at the same time. The basic unit of execution on a GPU (the compute shader or kernel) is run in parallel and scheduled by the GPU execution context
and the commands are setup and issued by the cpu application in a queue fashion, some execution can be synched to wait for other kernels to complete or can be dispatched and run in whatever order.


\subsubsection{Explanation of GPGPU Processing Models}~\\
The most basic unit of Processing on a GPU is work item (AMD) or thread (NVIDIA) this represents a single piece of data that is processed in parallel. Most GPUs organize these processors in groups called 
compute units which are controlled by a scheduler to process data in blocks on each compute unit. For an example such as modifying the pixel values of an image, each compute unit takes a part of the image 
and process it until the entire image is finished. The dimensions of the entire workload are set as a parameter to the GPU giving it the total number of elements to process as an N-Dimensional range (right 
now most GPUs go up to 3 Dimensions). This global size is then divided up by a work group size, and the work groups are then processed by the compute units \textbf{ (see figure \ref{fig:gpu})}. The dimensions of these work groups 
can be modified to optimize for different algorithms depending on how they access data. 
	\begin{figure}[H]
	\includegraphics[width=\linewidth,height=\paperheight,keepaspectratio]{gpgpu.jpg}
	\caption{GPGPU Diagram}
	%Figure \ref{fig:gpgpuImg} shows our blockDiagram.
	\label{fig:gpu}
	\end{figure}
\pagebreak
\subsubsection{Unity Support for Asynchronous Compute Shaders and Kenrels}
\paragraph{ Shading Languages used in Unity} ~\\
In Unity, shader programs are written in a variant of HLSL language (also called Cg but for most practical uses the two are the same). Unity recommneds developers have a good understanding 
of OpencL Or CUDA, OpenGL, and other Graphics/ GPGPU parallel compute languages.  

Internally, different shader compilers are used for shader program compilation:
\begin{itemize}
  \item Windows and Microsoft platforms (DX9, DX11, DX12, XboxOne and Xbox 360 ) all use Microsoft's HLSL compiler.
  \item OpenGL Core (GL3) and OpenGL ES 3 use Microsoft’s HLSL followed by bytecode translation into GLSL, using a modified version of hlslcc.
  \item OpenGL Legacy (GL2), OpenGL ES 2.0 and Metal use source level translation via hlsl2glslfork and glsl optimizer. 
  \item Other console platforms use their respective compilers (e.g. PSSL on PS4).
  \item Surface Shaders use Cg 2.2 and MojoShader for code generation analysis step.
\end{itemize} %unity ref http://docs.unity3d.com/Manual/SL-ShadingLanguage.html
\pagebreak
% try unity zip in google drive with atomic and local mem barriers
Some advanced GPGPU related constructs useful for highly optimized compute processing, need to be checked if they are included in Unity's implementation of DirectCompute,which 
is similar to the HLSL code snippets listed below.
\subsubsection{Reduction Kernel}
\begin{figure}[H]
	\label{fig:reductionCode}
\lstinputlisting [language =C] {shared_mem_reduction.hlsl}
\caption{Reduction Code}
\end{figure}

This kernel is a computeShader that is run on the GPU and allows for synchronization of shared memory across multiple compute units, this technique is useful for parallel reduction algorithms which include
calcualtions such as min, max, mean, sum etc... anything that can be reduced to simpler worksets are are not itterative in nature. \textbf{(see figure \ref{fig:reductionImg})}

\begin{figure}[H]
	\includegraphics[width=\linewidth,height=\paperheight,keepaspectratio]{reduction.jpg}
	\caption{Reduction Diagram}
	%Figure \ref{fig:gpgpuImg} shows our blockDiagram.
	\label{fig:reductionImg}
	\end{figure}
\subsubsection{Atomic Kernel}
\begin{figure}[H]
\lstinputlisting [language =C] {atomic_inc.hlsl}
This kernel is a computeShader that is run on the GPU and allows for synchronization adding or various operations that are done in an atomic fashion, avoiding, race conditions.
\caption{Atomic Code}
%Figure \ref{fig:gpgpuImg} shows our blockDiagram.
\label{fig:atomicCode}
\end{figure}

\pagebreak
%E. Explicit Design Summary with diagrams
%F. Build, prototype, test, and evaluation plan
\section{Design Documentation}
	\subsection{Block Diagrams:}
	\begin{figure}[H]
	\includegraphics[width=\linewidth,height=\paperheight,keepaspectratio]{HardwareConfig.png}
	\caption{Hardware Block Diagram}
	%to ref fig number
	%Figure \ref{fig:block1} shows our blockDiagram.
	\label{fig:hblock}
	\end{figure}
	This diagram represents a hardware configuration of a typical VR system that our application would expect. This may change depending on the device we use, but is a good overview.
	\pagebreak
	\begin{figure}[H]
	\includegraphics[width=\linewidth,height=\paperheight,keepaspectratio]{SoftwareConfig.png}
	\caption{Software Block Diagram}
	%to ref fig number
	%Figure \ref{fig:block2} shows our blockDiagram.
	\label{fig:sblock}
	\end{figure}
	This diagram represents the software design of our project and the various functional modules that should exist in order to fulfill the user interaction requirements. 
	The application will have two main modes of operation, edit mode and view mode. 
	\subsection{Qt Design Plan}
		\subsubsection{High Level Diagram}
		\subsubsection{Activity Diagram}
	\subsection{Unity Hierarchy}
		\subsubsection{High Level Design}
		\subsubsection{Activity Diagram}
		\subsubsection{Scene Shared Assets}
		
\pagebreak

%4. Administrative content
%G. Personnel and bibliography of related work, if any (mostly text)
%H. Facilities and Equipment (text, numbers, tables, charts, figures, diagrams)
%I. Consultants, subcontractors, and suppliers (mostly text)
%A. Budget and financing (text, numbers, tables, charts, figures, diagrams).
%B. Milestone chart for all activities related to the project
\section{Administrative Content}
\section{Budget and Financing}
	                    Financing is still being researched at this point.  Due to us not having a sponsor, we will either have to self-finance or contact people willing to help with funding.  Therefore, our budget and financing are extremely subject to change. We already have the hardware necessary to run virtual reality hardware, so that saves us the cost of building one ourselves (upwards of 1000 dollars). VR headsets vary in price, although our current target the Oculus Rift, costs 600 dollars. Using Unity Pro costs 75 dollars a month, so over the next 8 months it would add up to another 600 dollars. This totals to 1200 dollars for development and hardware. However, a number of on campus facilities also have virtual 
	                    reality headsets for testing and design purposes, including Sony’s VR headset Morpheus and a few Oculus Rifts.  If we can gain access to these facilities, we can cut down on hardware costs tremendously.  Additionally, if we can gain a sponsor or funding from UCF, we could potentially up our budget and invest in some newer VR 
	                    technology, and possibly be able to test for multiple platforms. The upper limit would be the Microsoft Hololens, which costs between 1500 dollars and 3000 dollars for a dev kit, while it is something we would 
	                    not be able to afford for this project alone, it could be something UCF would be interested in investing in.  
	                    
	                    \subsection{Summary:}
	                    Current Budget: Roughly \$1500.  \$600 for Dev Kit,
	                    \$600 for development Engine, \$200 for equipment and other software (in engine models, non-vr hardware such as controllers and cameras),\$100 for other
	                    expenses that could come up.
	                    
	                    \subsection{Funding}
	                    This is entirely self funded at the moment.  All of us have well-paying jobs in the 
	                    Orlando area, and would be able to spend \$500 if necessary.  However, hopefully we will be able to at the very least cut out the 
	                    \$600 Oculus Rift Dev Kit cost by using UCF facilities, and possibly cutting away the \$600 engine cost as well by settling on the limited free version of Unity.  
	%software licensing costs, cloud based service costs, code repo's, graphic design costs
	\section{Schedule}
	\begin{figure}[H]
	\includegraphics[width=\linewidth]{scheduleSR.png}
	\caption{Prototype Phase Gantt Chart:}
	%to ref fig number
	%Figure \ref{fig:block1} shows our blockDiagram.
	\label{fig:pchart}
	\end{figure}
	\section{Milestones}
	\subsection{Tech Milestones}
		\begin{itemize}
			\item Finding funding or available UCF facilities
			\item Obtaining Oculus Rift
			\item Obtaining Gear VR
			\item Obtaining LeapMotion
			\item Getting virtual reality hardware to display world
			\item Connecting user input to world
		\end{itemize}
	\subsection{Research Milestones}
		\begin{itemize}
			\item Choose the most important scenes to develop
			\item Testing with psychologists to confirm project is beneficial
		\end{itemize}
	\subsection{Development Milestones}
		\begin{itemize}
			\item Allowing for users to add models to world, and design said world
			\item Combining all aspects of project into cohesive whole
			\item Finalize and begin testing of project
		\end{itemize}
	\subsection{Overall Milestones}
		\begin{itemize}
			\item 
		\end{itemize}

\pagebreak
%5. Project Summary and conclusions.
\section{Project Summary and Conclusions}

\pagebreak
\pagenumbering{Alph}
\setcounter{page}{1}
%6. References
\section{References}

\bibliographystyle{mla}
\begin{thebibliography}{9}
\bibitem{stroke1}
Rettner, Rachael. "Stroke Therapy Gets Boost from Virtual Reality." LiveScience. TechMedia Network, 07 Apr. 2011. Web. 06 Apr. 2016.
\bibitem{dsmCaution}
"Cautionary Statement for DSM IV - TR." Cautionary Statement for DSM IV. N.p., n.d. Web. 10 Apr. 2016.
\bibitem {dsmPhobia}
"Diagnostic Criteria for 300.29 Specific Phobia." Diagnostic Criteria for 300.29 Specific Phobia. N.p., n.d. Web. 10 Apr. 2016.
\end{thebibliography}

%7. Appendices
\section{Appendices}

%A. Copyright permissions
\section{Copyright Permissions}

%B. Data-sheets (if necessary)
%C. Software (if necessary)
%D. Other
\section{Extras}

\end{document}
