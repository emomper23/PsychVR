
% binding info: NO NOTEBOOKS OR RING BINDERS, proefssional appearance, non-paper cover, bound. 8.5x11 w/ 1 inch margins, left side 1.5in margin.

% text formatting info: The paragraphs are to be fully justified (both left and right sides). New paragraphs may begin by indenting the line or by not indenting but leaving a space. However, DO NOT do both. The body font must be Times Roman, Arial, Helvetica, or be approved by the instructor with a font size of 10-12 pts. Heading fonts can be no larger than 20 pts. The document must be single-spaced and printing can be single or double sided. Color printing is optional and left to the discretion of the groups.

% sources info: 5. The appendix must contain written authorization (emails, letters, or explicit permission citations) for rights to include or use copyrighted content. 6. All copyrighted contents must display the content’s origin or author, and an appropriate phrase such as “reprinted with permission.” 

% citing info: 7. All elements of the document that support the written text, such as figures, table, illustrations, code segment, charts, etc., must be cited in the body of the text, and the citation must appear before the element is shown. Supportive elements cannot begin a chapter, section, or paragraph. All supportive material must be captioned, which include the element name and number and a description. If you do not author the element, the caption must contain wording identifying that you have permission to use the element. The actually permission authorization must be included in the appendices.

% reference format info: 8. Your document should contain a “References” section at the end that lists all of the source material cited in your document. The references should be listed in the order that they first appear in your document in a format like the following (first is a book, second is an article): [1] J. Hennessy and D. Patterson. Computer Architecture: A Quantitative Approach, 5 th edition, Morgan Kaufman publishers, 2014. [2] M. Heinrich et al. “The Performance Impact of Flexibility in the Stanford FLASH Mujltiprocessor”, In Proceedings of the 6th International Conference on Architectural Support for Programming Languages and Operating Systems (ASPLOS), pages 274-285, 1994

% citation formatting: When citing the references in the text use a non-breakable space (e.g. Ctrl-shift-space in Word, \&; in HTML) before the citation and then square brackets and the number: Some researchers say that the occupancy of the node controller is the parameter most critical to the performance of the overall machine [2]. If you want to cite multiple references at once just separate them by commas like so [1,3,7].

\documentclass[a4paper,10pt]{article}
\usepackage[utf8x]{inputenc} 
\usepackage{graphicx}
\usepackage{float}
\usepackage{geometry}
	\geometry{
	letterpaper,
	total={8.5in,11in},
	left=1.5in,
	top=1in,
	right=1in,
	bottom=1in
}

%opening
\pagenumbering{roman}
%1. Cover page with title, group number, team members, date, and any other relevant information, such as participating organizations and sponsors.
\title{Group 12 Project Final Design Document}
\author{Eric Momper, Peter Lomason, John Barber}
\setlength{\parindent}{0pt}
\begin{document}
	
	\maketitle
	
	\pagebreak
	\tableofcontents
	\pagebreak
	
	\section{Introduction}
		Our Project is intended to be a psychological theraputic tool that helps patients through simulation in Virtual Reality.
	Many of the new up and coming Virtual Reality devices look very promising at providing better and more realistic immersion and simulation for users at a lower cost than ever before. We aim to bring psychological tools to the home of the average user. This will allow those seeking certain therapies or treatments to perform them more often since such services often require high diligence and repetition to produce results.
	
	\paragraph{A few of these devices include:}
	\begin{itemize}
		\item Occulus Rift, Occulus Touch
		\item HTC Vive
		\item Samsung Gear VR
		\item Google Cardboard
		\item Microsoft Hololens
	\end{itemize}
	
	\paragraph{ Some of our project ideas are related to the following:}
	\begin{itemize}
		\item ​Immersion Therapy for phobias, dysphorias, or PTSD (Virtual or Augmented Reality)
		\item Therapy for burn victims, phantom pain for amputees,  (Virtual Reality)
		\item Creating a calm environment for  anxiety disorders, or autism (Virtual Reality)
		\item Creating a drawing art therapy tool that allows for creativity in 3D space (Virtual or Augmented Reality)
	\end{itemize}
	
	
	\paragraph{Project Direction} ~\\ We are currently contacting faculty in the psychology department for collaboration or guidance on opportunities to use our project for helping their patients or for their research.
	\paragraph{Integration} ~\\ Our project will be created in Unity, Unnreal Engine, or written with a C++ OpenGL/ DirectX SDK wrapper, depending on the direction of our design, and the VR device we have available. It will most likely be windows exclusive as many VR devices are dropping Linux and OSX support unless we can find some options that will provide cross platform systems, which is said to be available in the near future for some platforms but not currently finished.  
	
	\pagebreak
	
	\section{Status of Virtual Reality in Psychology}
	The first use of Virtual Reality therapy in Psychology was in 1995 by psychologist Barbara Rothbaum and computer scientist Larry Hodges. T
	hey found that virtual reality therapy could help patients overcome phobias such as arachnophobia or a fear of heights. Since then many others 
	have used virtual reality as a tool in psychology. The main use of virtual reality in psychology is a form of treatment called Exposure Therapy. 
	This type of treatment can be used to address psychological issues such as Autism spectrum disorder, Obsessive Compulsive Disorder, various phobias, 
	post-traumatic stress disorder, and phantom limb pain. The greatest issue facing treatment with Exposure Therapy is that it requires a high level of diligence and
	repetition. Typically patients cannot make time for appointments as frequently as is required.
	
	\subsection{Immersion Therapy}
	This psychological treatment helps patients simulate past or hypothetical events so they can adapt to or reason through
	various situations. 
	\subsection{Pain Treatment}
	Some examples of this are used with burn victims or people with acute pain, putting them in a calming or cold environment to distract them to relieve pain.
	\subsection{Creating a Calming Environment}
	This can be useful for many patients with anxiety disorders as it will allow users to be in a relaxing environment with softer stimuli that can take their mind off of anxiety issues. 
	\subsection{Art Therapy}
	This can be a creative environment where users can have 3D artistic expression where they can draw colors or shapes in 3D space. 
	Artistic expression has been proven to relieve stress and calm users.  
	\pagebreak
	
	
	\section{Team Member Motivations}
	\subsection{Eric Momper}
	My personal motivation for this project is that it is similar to some of the programming I do on my internship (OpenGL and OpenCL image GPGPU processing).
	I also have taken computer graphics with Professor Leinicker and I am currently taking robot vision with Doctor Lobo. This project will be very interesting to me as  
	I will be working with new technologies and programming on a new type of 3D graphics platform. I am also looking forward to studying the psychological benefits
	of Virtual Reality on patients with various disorders.  
	
	\subsection{John Barber}
	This project was my idea, and combines two different fields, virtual reality and psychology.  In the same way, i'm interested in both 
	sides separately, and really hopeful about how they can be combined.  Virtual Reality is exciting to me as a game designer, 
	and one of my closest friends went to work for Emblematic Group, and we've been comparing notes on the future of virtual reality since.  
	However, I also have seen a number of my friends struggle with psychological issues, and have been hoping to find a way to use my Computer 
	Science degree after I graduate to help people.  This is an opportunity to not only accomplish that now, pushing the field forward and finding 
	new ways to use it for people, but also to train myself and find opportunities and connections for the future.
	\subsection{Peter Lomason}
	My interest in this project mainly comes from past experience with virtual reality tools. I used to own an Oculus Rift DK1 and at the time I had it, 
	I was not knowledgeable enough to develop for it. Now I would like to apply what I have learned at UCF to virtual reality development. I eventually want to be
	a video game developer and virtual reality for psychology shares many aspects with that. Being able to program an environment, objects, and interactions is very
	interesting to me and this project will allow me to strengthen my ability to do these tasks.
\pagebreak



\pagenumbering{arabic}

%Project Documentation Guidelines
%additions / deletions where appropriate

%2. Executive summary: An administrative and technical abstract, which includes a brief description of the project, the project objectives, and the technical approach. This is really an overview of 3A, 3B, 3C, and 3D. This is page number 1.
\section{Executive Summary}

\pagebreak
%3. Technical content (This is NOT an outline, just a list of what needs to be included)
%A. Identification of the project and its significance, motivation, etc. (mostly text). Please include personal motivation statements for each project member here. Also you MUST include a separately labeled section or subsection with the title “Broader Impacts” that describes in a minimum of 1 paragraph, the broader implications or impact of your project on society both local and global. How might your project impact underrepresented groups (within science and technology (STEM) or society as a whole), the disabled, non-profit organizations, the environment, diversity, increased participation in STEM fields or the workforce, public engagement in STEM areas, improved national security, enhanced infrastructure, or improved education are all examples.
%B. Technical objectives, goals, specifications, and requirements (mostly text and numbers) FIRST PAPER HERE EZPZ
\section{Project Goals and Objectives}
	Design an environment in Virtual Reality that can be customized and used to help a variety of psychological conditions, and further psychology research, that is accessible to the typical user so they may conduct their own treatment on their own time.
	\begin{itemize}
		\item Increase our understanding of psychology principals \& problems and how virtual reality can help certain conditions.
		\item Find a platform that we can develop on, that creates a high quality virtual reality experience, and is reasonably up to date with modern graphics. Currently our best candidate is the Unity Engine.
		\item Integrate some level of user intractability in the created virtual environments. 
		\item Procedural generation of environments with parameters that can be customized by the user. 
		\item Include pre-constructed environments similar to current treatment strategies in the VR Psychology industry.
	\end{itemize}
	\section{Broader Impacts}
	%Broad implications and impact on society (impact on underrepresented? within STEM and or society as a whole? disabled? non-profit orgs? environment? diversity? increased participation in STEM or workforce? public engagement in STEM? improve national security? enhanced infrastructure? improved education?)
	
	%qualitative, avoid numbers, conceptual discussion specific to project. example descriptions "“lightweight, portable, programmable, low cost, flexible, high resolution, scalable, low power, accurate, mobile, peer-to-peer, autonomic”
	
	Our project would have an impact on the field of psychology as it relates to certain emotional disorders or struggles and the deployment of self-administered therapy for those disorders. It would not only be relevant to those diagnosed with a psychological condition, but also those seeking stress relief or a unique environment to immerse themselves in. By bringing treatment therapies to the end user in their own home and allowing them to perform therapy at their leisure we will have an impact on many people seeking these services who may not have the time to schedule appointments or those who simply won't try due to the stigma surrounding therapy.
	\pagebreak
	\section{Function}
	Make a customizable environment which includes settings for some or all of following:
	\subsection{Therapeutic Art Deign Environment}
	This calming art suite allows users to relax and design art in a creative 
	\begin{itemize}
		\item The user can use hand controls or a controller to draw colors and shapes in a 3D space around them. 
		\item This can be used on a variety of platforms including Ouculus Rift, HTC Vive, and PS VR.
	\end{itemize}   
	\subsection{Music Visualization}
	This module would be related to music visualization. 
	\begin{itemize}
		\item The different sound or frequencies that could be visualized via colors and waves that are around the user as music is playing in real time.
		\item The user can interact with these sounds in some way possible a puzzle.(see auditorium game)
	\end{itemize}   
	\subsection{Virtual 3D Worlds Specific to Various Conditions}
	This module would consist of a variety of 3D worlds that the user could move in and interact with. 
	\begin{itemize}
		\item Environments will have some sort of editable creative mode interface where some customization is allowed and saved.
		\item The user is loaded into a 3D pre-designed environment.
		\item The user can use hand controls to interact with the created environment and move around. 
		\item This can be used on a variety of platforms including Ouculus Rift, HTC Vive, and PS VR
	\end{itemize}
\pagebreak
%D. Detailed design content (text, numbers, tables, charts, figures, diagrams)
\section{Specifications and Requirements}
\subsection{Functional Requirements}
	\begin{enumerate}
		\item Project will create some 3D environment that the user can see and interact with in 3D space.
		\item The project must have modules that provide some kind of psychological treatment for end users at home.
		\item The pre-built environments will allow users to modify and change certain settings then save those preferences.
		\item There will be controls to allow the user to interact with menus and the environment as well as make modifications or adjustments.
	\end{enumerate}
	
	\subsection{Performance Requirements}
	\begin{enumerate}
		\item User must have a computer that runs windows (for now) and has sufficient GPU and other system specs to run a VR headset.
		\item The VR environment must run at least 30 FPS on an appropriate platform.
		\item User control interactions must be responsive and fluid (reasonable interaction responses).
	\end{enumerate}
	\subsection{Users and Human Factors Requirements}
	\begin{enumerate}
		\item The dialogs and menus must be easy to see and use.
		\item Some user interactions must be desktop based, rather than within VR, to be simpler (ex file menus).
		\item The models, textures, and graphics in the environment must be decent quality to help immerse the user.
	\end{enumerate}
	
\pagebreak
%C. Research and investigations (text, numbers, tables, charts, figures, diagrams)
\section{Research}

\pagebreak
%E. Explicit Design Summary with diagrams
%F. Build, prototype, test, and evaluation plan
\section{Design Documentation}

\pagebreak
%4. Administrative content
%G. Personnel and bibliography of related work, if any (mostly text)
%H. Facilities and Equipment (text, numbers, tables, charts, figures, diagrams)
%I. Consultants, subcontractors, and suppliers (mostly text)
%A. Budget and financing (text, numbers, tables, charts, figures, diagrams).
%B. Milestone chart for all activities related to the project
\section{Administrative Content}
\section{Budget and Financing}
	                    Financing is still being researched at this point.  Due to us not having a sponsor, we will either have to self-finance or contact people willing to help with funding.  Therefore, our budget and financing are extremely subject to change. We already have the hardware necessary to run virtual reality hardware, so that saves us the cost of building one ourselves (upwards of 1000 dollars). VR headsets vary in price, although our current target the Oculus Rift, costs 600 dollars. Using Unity Pro costs 75 dollars a month, so over the next 8 months it would add up to another 600 dollars. This totals to 1200 dollars for development and hardware. However, a number of on campus facilities also have virtual 
	                    reality headsets for testing and design purposes, including Sony’s VR headset Morpheus and a few Oculus Rifts.  If we can gain access to these facilities, we can cut down on hardware costs tremendously.  Additionally, if we can gain a sponsor or funding from UCF, we could potentially up our budget and invest in some newer VR 
	                    technology, and possibly be able to test for multiple platforms. The upper limit would be the Microsoft Hololens, which costs between 1500 dollars and 3000 dollars for a dev kit, while it is something we would 
	                    not be able to afford for this project alone, it could be something UCF would be interested in investing in.  
	                    
	                    \subsection{Summary:}
	                    Current Budget: Roughly \$1500.  \$600 for Dev Kit,
	                    \$600 for development Engine, \$200 for equipment and other software (in engine models, non-vr hardware such as controllers and cameras),\$100 for other
	                    expenses that could come up.
	                    
	                    \subsection{Funding}
	                    This is entirely self funded at the moment.  All of us have well-paying jobs in the 
	                    Orlando area, and would be able to spend \$500 if necessary.  However, hopefully we will be able to at the very least cut out the 
	                    \$600 Oculus Rift Dev Kit cost by using UCF facilities, and possibly cutting away the \$600 engine cost as well by settling on the limited free version of Unity.  
	%software licensing costs, cloud based service costs, code repo's, graphic design costs
	\section{Schedule}
	\begin{figure}[H]
	\includegraphics[width=\linewidth]{scheduleSR.png}
	\caption{Prototype Phase Gantt Chart:}
	%to ref fig number
	%Figure \ref{fig:block1} shows our blockDiagram.
	\label{fig:pchart}
	\end{figure}
	\section{Milestones}
	\begin{itemize}
	  \item Finding funding or available UCF facilities
\item Creating basic 3d world
\item Getting virtual reality hardware to display world
\item Connecting user input to world
\item Allowing for users to add models to world, and design said world
\item Testing with psychologists to confirm project is beneficial
\item Combining all aspects of project into cohesive whole
\item Finalize and begin testing of project
	\end{itemize}


\pagebreak
%5. Project Summary and conclusions.
\section{Project Summary and Conclusions}

\pagebreak
\pagenumbering{Alph}
\setcounter{page}{1}
%6. References
\section{References}

%7. Appendices
\section{Appendices}

%A. Copyright permissions
\section{Copyright Permissions}

%B. Data-sheets (if necessary)
%C. Software (if necessary)
%D. Other
\section{Extras}

\end{document}
